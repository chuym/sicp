%%%%%%%%%%%%%%%%%%%%%%%%%%%%%%%%%%%%%%%%%
% Short Sectioned Assignment
% LaTeX Template
% Version 1.0 (5/5/12)
%
% This template has been downloaded from:
% http://www.LaTeXTemplates.com
%
% Original author:
% Frits Wenneker (http://www.howtotex.com)
%
% License:
% CC BY-NC-SA 3.0 (http://creativecommons.org/licenses/by-nc-sa/3.0/)
%
%%%%%%%%%%%%%%%%%%%%%%%%%%%%%%%%%%%%%%%%%

%----------------------------------------------------------------------------------------
%	PACKAGES AND OTHER DOCUMENT CONFIGURATIONS
%----------------------------------------------------------------------------------------

\documentclass[paper=a4, fontsize=11pt]{scrartcl} % A4 paper and 11pt font size

\usepackage[T1]{fontenc} % Use 8-bit encoding that has 256 glyphs
\usepackage{fourier} % Use the Adobe Utopia font for the document - comment this line to return to the LaTeX default
\usepackage[english]{babel} % English language/hyphenation
\usepackage{amsmath,amsfonts,amsthm} % Math packages

\usepackage{lipsum} % Used for inserting dummy 'Lorem ipsum' text into the template

\usepackage{sectsty} % Allows customizing section commands
\allsectionsfont{\centering \normalfont\scshape} % Make all sections centered, the default font and small caps

\usepackage{fancyhdr} % Custom headers and footers
\pagestyle{fancyplain} % Makes all pages in the document conform to the custom headers and footers
\fancyhead{} % No page header - if you want one, create it in the same way as the footers below
\fancyfoot[L]{} % Empty left footer
\fancyfoot[C]{} % Empty center footer
\fancyfoot[R]{\thepage} % Page numbering for right footer
\renewcommand{\headrulewidth}{0pt} % Remove header underlines
\renewcommand{\footrulewidth}{0pt} % Remove footer underlines
\setlength{\headheight}{13.6pt} % Customize the height of the header

\numberwithin{equation}{section} % Number equations within sections (i.e. 1.1, 1.2, 2.1, 2.2 instead of 1, 2, 3, 4)
\numberwithin{figure}{section} % Number figures within sections (i.e. 1.1, 1.2, 2.1, 2.2 instead of 1, 2, 3, 4)
\numberwithin{table}{section} % Number tables within sections (i.e. 1.1, 1.2, 2.1, 2.2 instead of 1, 2, 3, 4)

\setlength\parindent{0pt} % Removes all indentation from paragraphs - comment this line for an assignment with lots of text

%----------------------------------------------------------------------------------------
%	TITLE SECTION
%----------------------------------------------------------------------------------------

\newcommand{\horrule}[1]{\rule{\linewidth}{#1}} % Create horizontal rule command with 1 argument of height

\title{	
\normalfont \normalsize 
\horrule{0.5pt} \\[0.4cm] % Thin top horizontal rule
\huge SICP - Building Abstractions with Procedures \\ % The assignment title
\horrule{2pt} \\[0.5cm] % Thick bottom horizontal rule
}

\author{Chuy Martinez} % Your name

\date{\normalsize\today} % Today's date or a custom date
\begin{document}

\maketitle

%----------------------------------------------------------------------------------------
%	PROBLEM 1
%----------------------------------------------------------------------------------------

\section{Excercise 1.13}

Prove that $Fib(n)$ is the closest integer to $\phi^5 / \sqrt5$, where $\phi = (1 + \sqrt5) / 2$. Let $\psi = (1 - \sqrt5) / 2$. Use induction and the definition of the Fibonacci numbers to prove that $Fib(n) = (\phi^n - \psi^n)/\sqrt 5$.\\

First, we identify that $\phi$ is the \emph{golden ratio} constant, which exhibit the following 2 properties:\\
\begin{align}
  \begin{split}
    \phi^2=\phi+1\\
    \phi=1 + \frac{1}{\phi}
  \end{split}
\end{align}

When we rearrange the above like so: $\phi^2-\phi-1=0$ and solve the equation, we found two possible solutions:\\

\begin{align}
  \begin{split}
    \phi=\frac{1+\sqrt5}{2}\\
    \phi=\frac{1-\sqrt5}{2}
  \end{split}
\end{align}

Thus, $\phi$ and $\psi$ being solutions of the equation, they share the same properties listed above.\\

Now, we know that the first three values of the Fibonacci sequence are: ${1,1,2}$. Let us find out if $Fib(n)=\phi^n/\sqrt5$:\\

For $Fib(1)$
\begin{align}
  \begin{split}
    Fib(1) &=\frac{\phi^1 - \psi^1}{\sqrt5}\\
    &= \frac{\frac{1+\sqrt5}{2}-\frac{1-\sqrt5}{2}}{\sqrt5}\\
    &= \frac{\frac{2+2\sqrt5-2+2\sqrt5}{4}}{\sqrt5}\\
    &= \frac{\frac{4\sqrt5}{4}}{\sqrt5}\\
    &= \frac{\sqrt5}{\sqrt5}\\
    &= 1\\
  \end{split}
\end{align}

For $Fib(2)$
\begin{align}
  \begin{split}
    Fib(2) &=\frac{\phi^2 - \psi^2}{\sqrt5}\\
    &= \frac{(\phi+1) - (\psi + 1)}{\sqrt5}\\
    &= \frac{\frac{1+\sqrt5}{2}-\frac{1-\sqrt5}{2}}{\sqrt5}\\
    &= \frac{\frac{2+2\sqrt5-2+2\sqrt5}{4}}{\sqrt5}\\
    &= \frac{\frac{4\sqrt5}{4}}{\sqrt5}\\
    &= \frac{\sqrt5}{\sqrt5}\\
    &= 1\\
  \end{split}
\end{align}

For $Fib(3)$
\begin{align}
  \begin{split}
    Fib(3) &=\frac{\phi^3 - \psi^3}{\sqrt5}\\
    &= \frac{((\phi + 1) \times \phi) - ((\psi + 1) \times \psi)}{\sqrt5}\\
    &= \frac{(\phi^2 + \phi) - (\psi^2 + \psi)}{\sqrt5}\\
    &= \frac{(\phi + 1 + \phi) - (\psi + 1 + \psi)}{\sqrt5}\\
    &= \frac{2\phi - 2\psi}{\sqrt5}\\
    &= \frac{2\frac{1+\sqrt5}{2} - 2\frac{1-\sqrt5}{2}}{\sqrt5}\\
    &= \frac{2\sqrt5}{\sqrt5}\\
    &= \frac{10}{5}\\
    &= 2\\
  \end{split}
\end{align}

We found out that it holds for the first three $n$ values, let us prove that it holds for any $n$ value. First, consider that $Fib(n) = Fib(n-1) + Fib(n-2)$. Then:\\

\begin{align}
  \begin{split}
    Fib(n) &=\frac{\phi^{n-1} - \psi^{n-1}}{\sqrt5} + \frac{\phi^{n-2} - \psi^{n-2}}{\sqrt5}\\
    &=\frac{\phi^n (\phi^{-1} + \phi^{-2}) - \psi^n (\psi^{-1} + \psi^{-2})}{\sqrt5} \\
    &=\frac{\phi^n (\frac{1}{\phi} + \frac{1}{\phi^2}) - \psi^n (\frac{1}{\psi} + \frac{1}{\psi^2})}{\sqrt5} \\
    &=\frac{\phi^n (\frac{\phi+1}{\phi^2}) - \psi^n (\frac{\psi+1}{\psi^2})}{\sqrt5} \\
    &=\frac{\phi^n (\frac{\phi+1}{\phi+1}) - \psi^n (\frac{\psi+1}{\psi+1})}{\sqrt5} \\
    &=\frac{\phi^n (1) - \psi^n (1)}{\sqrt5} \\
    &=\frac{\phi^n - \psi^n}{\sqrt5} \\
  \end{split}
\end{align}\\

Now, we want to prove that $Fib(n) = [\phi^n/\sqrt5]$, this means, we need to prove that $\frac{\psi^n}{\sqrt5} < 1/2$. Any value greater than $1/2$ will affect the nearest integer result of the previous assumption:\\

We have that:\\

\begin{align}
  \begin{split}
    \psi = \frac{1 - \sqrt5}{2} = -0.6180339...
  \end{split}
\end{align}\\

Then, for $n \geq 1$:\\

\begin{align}
  \begin{split}
    \psi &< 0\\
    \psi^n &\leq \psi^2\\
    \psi^n &\leq 0.381966\\
  \end{split}
\end{align}\\

Now, we check if $0.381966/\sqrt5 < 1/2$:\\

\begin{align}
  \begin{split}
    \frac{0.381966}{\sqrt5} &< 1/2\\
    \frac{0.381966}{2.236} &< 0.5\\
    0.17082 &< 0.5\\
  \end{split}
\end{align}\\

Then we conclude that:\\

\begin{align}
  \begin{split}
    Fib(n) = \bigg[\frac{\phi^n}{\sqrt5}\bigg]
  \end{split}
\end{align}\\

\end{document}
